\documentclass[]{tufte-handout}

% ams
\usepackage{amssymb,amsmath}

\usepackage{ifxetex,ifluatex}
\usepackage{fixltx2e} % provides \textsubscript
\ifnum 0\ifxetex 1\fi\ifluatex 1\fi=0 % if pdftex
  \usepackage[T1]{fontenc}
  \usepackage[utf8]{inputenc}
\else % if luatex or xelatex
  \makeatletter
  \@ifpackageloaded{fontspec}{}{\usepackage{fontspec}}
  \makeatother
  \defaultfontfeatures{Ligatures=TeX,Scale=MatchLowercase}
  \makeatletter
  \@ifpackageloaded{soul}{
     \renewcommand\allcapsspacing[1]{{\addfontfeature{LetterSpace=15}#1}}
     \renewcommand\smallcapsspacing[1]{{\addfontfeature{LetterSpace=10}#1}}
   }{}
  \makeatother
\fi

% graphix
\usepackage{graphicx}
\setkeys{Gin}{width=\linewidth,totalheight=\textheight,keepaspectratio}

% booktabs
\usepackage{booktabs}

% url
\usepackage{url}

% hyperref
\usepackage{hyperref}

% units.
\usepackage{units}


\setcounter{secnumdepth}{-1}

% citations
\usepackage{natbib}
\bibliographystyle{plainnat}

% pandoc syntax highlighting

% longtable

% multiplecol
\usepackage{multicol}

% strikeout
\usepackage[normalem]{ulem}

% morefloats
\usepackage{morefloats}


% tightlist macro required by pandoc >= 1.14
\providecommand{\tightlist}{%
  \setlength{\itemsep}{0pt}\setlength{\parskip}{0pt}}

% title / author / date
\title{Analyzing a Scientific Paper}
\author{Craig W. Whippo}
\date{2017-09-15}


\begin{document}

\maketitle




\section{1. Introduction Section}\label{introduction-section}

The introdution section of a well written scientific text needs to
accomplish three things. First, it needs to establish the general scope
of the field and review what is known. This background knowledge gives
readers enough contextual information to understand the research
article. Second, the introduction section should explicity state reasons
why additional research is needed.

\subsection{A. Background knowledge}\label{a.-background-knowledge}

\subsection{B. Motive}\label{b.-motive}

Why is this research important? Why is additional information needed?

\subsection{C. Objective}\label{c.-objective}

What did the scientists want to learn by doing this study?

\subsection{D. General Approach}\label{d.-general-approach}

\section{2. Results Section}\label{results-section}

\subsection{A. Objectives}\label{a.-objectives}

\subsection{B. Approach}\label{b.-approach}

\subsection{C. Organization of the
Inscriptions}\label{c.-organization-of-the-inscriptions}

\subsection{D. Interpretation of the
Inscriptions}\label{d.-interpretation-of-the-inscriptions}

\subsubsection{i. Literal
Interpretations}\label{i.-literal-interpretations}

\subsubsection{ii. Authors'
Interpretations}\label{ii.-authors-interpretations}

\subsubsection{iii. Statement of
Finding}\label{iii.-statement-of-finding}

\section{3. Discussion Section}\label{discussion-section}

\subsection{A. Main conclusions}\label{a.-main-conclusions}

Was the study able to

\subsection{B. Implications}\label{b.-implications}

\subsection{C. Support}\label{c.-support}

\subsection{D. Counterarguement}\label{d.-counterarguement}

\subsection{E. Refutation}\label{e.-refutation}

\bibliography{bib}



\end{document}
